%%%%%%%%%%%%%%%%%%%%%%%%%%%%%%%%%%%%%%%%%
% File name: US-MexCategorification2016.tex
% Poster name: Khovanov's Heisenberg category and the asymptotic representation theory of symmetric groups
% 
%
% Created by: Henry Kvinge
% Based on a template by:
% Computational Physics and Biophysics Group, Jacobs University
% https://teamwork.jacobs-university.de:8443/confluence/display/CoPandBiG/LaTeX+Poster
% 
% Further modified by:
% Nathaniel Johnston (nathaniel@njohnston.ca)
%
% This template has been downloaded from:
% http://www.LaTeXTemplates.com
%
% License:
% CC BY-NC-SA 3.0 (http://creativecommons.org/licenses/by-nc-sa/3.0/)
%
%%%%%%%%%%%%%%%%%%%%%%%%%%%%%%%%%%%%%%%%%

%----------------------------------------------------------------------------------------
%	PACKAGES AND OTHER DOCUMENT CONFIGURATIONS
%----------------------------------------------------------------------------------------

\documentclass[final]{beamer}
\usepackage[scale=1.24]{beamerposter} % Use the beamerposter package for laying out the poster
\usepackage{multicol, xcolor, tikz, dsfont,mathabx,mathtools}
\usepackage{multicol}
\usepackage{graphicx}

\usetheme{confposter} % Use the confposter theme supplied with this template


\setbeamercolor{block title}{fg=ngreen,bg=white} % Colors of the block titles
\setbeamercolor{block body}{fg=black,bg=white} % Colors of the body of blocks
\setbeamercolor{block alerted title}{fg=white,bg=dblue!70} % Colors of the highlighted block titles
\setbeamercolor{block alerted body}{fg=black,bg=dblue!10} % Colors of the body of highlighted blocks
% Many more colors are available for use in beamerthemeconfposter.sty

%-----------------------------------------------------------
% Define the column widths and overall poster size
% To set effective sepwid, onecolwid and twocolwid values, first choose how many columns you want and how much separation you want between columns
% In this template, the separation width chosen is 0.024 of the paper width and a 4-column layout
% onecolwid should therefore be (1-(# of columns+1)*sepwid)/# of columns e.g. (1-(4+1)*0.024)/4 = 0.22
% Set twocolwid to be (2*onecolwid)+sepwid = 0.464
% Set threecolwid to be (3*onecolwid)+2*sepwid = 0.708

\newlength{\sepwid}
\newlength{\onecolwid}
\newlength{\middlecolwid}
\setlength{\paperwidth}{48in} % A0 width: 46.8in
\setlength{\paperheight}{36in} % A0 height: 33.1in
\setlength{\sepwid}{0.024\paperwidth} % Separation width (white space) between columns
\setlength{\onecolwid}{0.3413\paperwidth} % Width of one column
\setlength{\middlecolwid}{0.22\paperwidth} % Width of two columns
\setlength{\topmargin}{-0.5in} % Reduce the top margin size
%-----------------------------------------------------------

%\usepackage{graphicx}  % Required for including images

\usepackage{booktabs} % Top and bottom rules for tables

%------------------------------- Tikz Related  ------------------------------------------------------------------------------------------

\usetikzlibrary{decorations.pathreplacing,shapes}
\usepackage{tikz}


%picture of clockwise bubble
\newcommand{\ckpicture}{
\begin{tikzpicture}
\draw[ultra thick] (0,0) circle (3cm);
\node at (-3,0) {\arrowlines};
\draw[fill=black] (-2.4,1.8) circle (.2cm);
\node at (-3,2.9) {$k$};
\end{tikzpicture}}

%picture of counterclockwise bubble
\newcommand{\ctildekpicture}{
\begin{tikzpicture}
\draw[ultra thick] (0,0) circle (3cm);
\node[rotate = 180] at (-3,0) {\arrowlines};
\draw[fill=black] (-2.4,1.8) circle (.2cm);
\node at (-3,2.9) {$k$};
\end{tikzpicture}}

%----------------------------------------------------------------------------------------
%	Omit command 
%----------------------------------------------------------------------------------------

\newcommand{\omitt}[1]{}

%----------------------------------------------------------------------------------------
%	TITLE SECTION 
%----------------------------------------------------------------------------------------

\title{Face Recognition via Principle Component Analysis} % Poster title

\author{Author: Jingya Li, Yujia Chen Instructor: Henry Kvinge} % Author(s)

%----------------------------------------------------------------------------------------

\begin{document}

\addtobeamertemplate{block end}{}{\vspace*{2ex}} % White space under blocks
\addtobeamertemplate{block alerted end}{}{\vspace*{2ex}} % White space under highlighted (alert) blocks

\setlength{\belowcaptionskip}{2ex} % White space under figures
\setlength\belowdisplayshortskip{2ex} % White space under equations

\begin{frame}[t] % The whole poster is enclosed in one beamer frame

\begin{columns}[t] % The whole poster consists of three major columns, the second of which is split into two columns twice - the [t] option aligns each column's content to the top

\begin{column}{\sepwid}\end{column} % Empty spacer column

\begin{column}{\onecolwid} % The first column

%----------------------------------------------------------------------------------------
%	Objective
%----------------------------------------------------------------------------------------

\setbeamercolor{block alerted title}{fg=black,bg=ngreen} % Change the alert block title colors
\setbeamercolor{block alerted body}{fg=black,bg=white} % Change the alert block body colors
\begin{alertblock}{Objective}
Use the singular value decomposition for a matrix to efficiently recognize human faces.
\end{alertblock}

\vspace{-4mm}

%----------------------------------------------------------------------------------------
%	Khovanov's Heisenberg category
%----------------------------------------------------------------------------------------

\begin{block}{Introduction}
Euclidean distance is an obvious way to try and distinguish between different human faces. Calculating the Euclidean distance between two length $kn$ vectors (corresponding to two $k \times n$ images), requires $\Big(kn\Big)^2$ subtractions and multiplications respectively. Using Matlab, the image:
\begin{figure}[h]
\centering
\includegraphics[width=.25\textwidth]{/Users/lijingya/Desktop/mathematics_project/pictures/sample_img.png}
\end{figure}
can be treated a 192$\times$168 matrix:\\
\begin{figure}[h]
\centering
\includegraphics[width=.3\textwidth]{/Users/lijingya/Desktop/mathematics_project/pictures/sample_img_size.png}
\end{figure}
Thus calculating the Euclidean distance for any two images of these dimensions requires $192\times168=32256$ multiplications and subtractions. Therefore, if the important features of a large collection of images can be summarized by just a few images, the computation cost can be reduced and the recognition speed can be improved. This can be achieved by PCA (Principle Component Analysis).
\end{block}

\vspace{-5mm}

%----------------------------------------------------------------------------------------
%	Subalgebra of symmetric functions
%----------------------------------------------------------------------------------------

\begin{block}{Mathematics of PCA}
Suppose we have $M$ vectors of size $N$ representing a set of sampled images. $p_j$'s represent the pixel values.
\begin{equation*}
x_i = [p_1,p_2,...,p_{j=N}]^T, \;\; \text{for } i=1,2,...,M.
\end{equation*}
Let $m$ be the pixelwise mean of the images
%\begin{equation*}
%m={1\over M}\sum_{i=1}^M x_i
%\end{equation*}
and let $w_i$ be defined as the mean centered image corresponding to $x_i$:
\begin{equation*}
w_i = x_i - m.
\end{equation*}
Find a set of vectors $\Big\{e_i\Big\}$ which best capture the variance of $w_1, \dots, w_M$. Specifically set
\begin{equation*}
e_1 := \text{arg max}_{||e||=1} \Big(\sum_{i}^M \Big\langle w_i, e\Big\rangle^2 \Big)
\end{equation*}
\end{block}

\end{column} % End of the first column

%--------------------------------------------------------------------------------------------
%Break between first and second column
%--------------------------------------------------------------------------------------------

\begin{column}{\sepwid}\end{column} % Empty spacer column

\begin{column}{\middlecolwid} % Begin second column

\begin{alertblock}{Main Result} 
\begin{figure}[h]
\centering
\includegraphics[width=.18\textwidth]{/Users/lijingya/Desktop/mathematics_project/pictures/person1.png}
\includegraphics[width=.18\textwidth]{/Users/lijingya/Desktop/mathematics_project/pictures/person2.png}
\includegraphics[width=.18\textwidth]{/Users/lijingya/Desktop/mathematics_project/pictures/person3.png}
\caption{ Faces from training set.}
\end{figure}
\begin{figure}[h]
\centering
\includegraphics[width=.17\textwidth]{/Users/lijingya/Desktop/mathematics_project/pictures/meanface.png}
\caption{ The average face after mean centering.}
\end{figure}
\begin{figure}[h]
\centering
\includegraphics[width=.18\textwidth]{/Users/lijingya/Desktop/mathematics_project/pictures/eigenperson1.png}
\includegraphics[width=.18\textwidth]{/Users/lijingya/Desktop/mathematics_project/pictures/eigenperson2.png}
\includegraphics[width=.18\textwidth]{/Users/lijingya/Desktop/mathematics_project/pictures/eigenperson3.png}
\caption{ The corresponded eigenfaces.}
\end{figure}
\begin{figure}[h]
\centering
\includegraphics[width=.29\textwidth]{/Users/lijingya/Desktop/mathematics_project/pictures/testimage.png}
\includegraphics[width=.29\textwidth]{/Users/lijingya/Desktop/mathematics_project/pictures/recognized_image.png}
\caption{ Test face vs. recognized face.}
\end{figure}
\begin{figure}[h]
\centering
\includegraphics[width=.29\textwidth]{/Users/lijingya/Desktop/mathematics_project/pictures/img_mat.png}
\includegraphics[width=.29\textwidth]{/Users/lijingya/Desktop/mathematics_project/pictures/project_mat.png}
\caption{ The reduction in computation.}
As we explained in the Introduction, by using PCA we were able to summarize the discriminatory features of a large collection of images of an individual in just a few images.
\end{figure}
\vspace{2mm}

\end{alertblock}

\vspace{-5mm}

\begin{block}


While in general 
\begin{equation*}
e_k = := \text{arg max}_{||e||=1} \Big(\sum_{i}^M \Big\langle \hat{w_i}, e\Big\rangle^2 \Big)
\end{equation*}
where 
\begin{equation*}
\hat{w_i} = w_i - \sum_{i =1}^{k-1} w_i e_i e_i^T.
\end{equation*}


\end{block}
\end{column} % End of the second column

%----------------------------------------------------------------------------------------
%	Third column
%----------------------------------------------------------------------------------------

\begin{column}{\sepwid}\end{column} % Empty spacer column

\begin{column}{\onecolwid} % Begin the third column

%----------------------------------------------------------------------------------------
%	Finishing shifted symmetric functions
%----------------------------------------------------------------------------------------


%----------------------------------------------------------------------------------------
%	Transition and co-transition measure
%----------------------------------------------------------------------------------------

\begin{block}

In other words, we repeatedly extract the direction which captures the highest variance in the data. This both acts as a tool for dimensionality reduction and also feature extraction since it efficiently captures variation in the data in a very compact form.

Given a new image $v_i$ (which we treat as a vector) we can calculate the angle between $v_i$ and the subspace spanned by $e_1, e_2, \dots, e_\ell$ for small $\ell$. The size of this angle should efficiently indicate the similarity of the new image to the set $x_1, \dots, x_M$.

\end{block}

\begin{block}{The Algorithm}
\begin{figure}[h]
\centering
\includegraphics[width=.8\textwidth]{/Users/lijingya/Desktop/mathematics_project/pictures/FC.png}
\caption{ The flow chart.}
\begin{itemize}
\item[i.] The feature extractor calculates the eigenfaces thus converting discriminatory information for each persons face to a small set of column vectors. 
\item[ii.] The test image is also converted to a vector. By comparing with a subset of the eigenface vectors for each person, we can predict which individual the test vector image corresponds to.

\end{itemize}
\end{figure}

\end{block}

\vspace{-8mm}

\omitt{
%----------------------------------------------------------------------------------------
%	Acknowledgements
%----------------------------------------------------------------------------------------

\setbeamercolor{block title}{fg=red,bg=white} % Change the block title color

\begin{block}{Acknowledgements}

\end{block}

\vspace{-8mm}}

%----------------------------------------------------------------------------------------
%	REFERENCES
%----------------------------------------------------------------------------------------

\begin{block}{References}
\begin{itemize}
\item The Yale Face Database B,\\ \url{http://vision.ucsd.edu/leekc/ExtYaleDatabase/ExtYaleB.html}
\end{itemize}
%\nocite{*} % Insert publications even if they are not cited in the poster
%\small{\bibliographystyle{unsrt}


\end{block}

\end{column} % End the third column

\end{columns} % End of all the columns in the poster

\end{frame} % End of the enclosing frame

\end{document}
